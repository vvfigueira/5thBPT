\documentclass[a4paper, 12pt]{article}
\usepackage[a4paper,
    left=1.5cm,
    right=1.5cm,
    top=1.5cm,
    bottom=1.5cm]{geometry}
\usepackage[font=small,labelfont=bf,
   justification=justified,
   format=plain]{caption}
\makeindex
\usepackage[brazilian]{babel}
\usepackage{amsthm}
\usepackage{graphicx}
\usepackage{setspace}
\usepackage{amsmath}
\usepackage{physics}
\usepackage{amssymb}
\usepackage{hyperref}
\usepackage{cleveref}
\usepackage{float}

\doublespacing

\title{\textbf{Imãs girando}}

% \author{\textbf{---}}

\newcommand{\cqd}{$\hfill\blacksquare$}
\newcommand{\hvb}[1]{\vb{\hat{#1}}}

\begin{document}

\maketitle

\section{Equações do movimento do caso exato}

O campo magnético gerado por um momento magnético $\vb m$ girando com velocidade angular $\omega$ no plano $xy$ é, sendo a posição do imã 
livre, $\vb r\qty(r, \theta, \phi)$, e a orientação de seu momento magnético, $\vb m'\qty(\theta', \phi', \psi')$,
$$\vb m = m \qty(\hvb x\cos\omega t+\hvb y\sin\omega t),\ \vb B = \frac{\mu_0}{4\pi r^3}\qty[3\qty(\vb m \cdot\hvb r)\hvb r-\vb m]$$
\begin{align}
    \vb B &= \frac{\mu_0 m}{4\pi r^3}\qty[3\sin\theta\cos\qty(\phi -\omega t)\mqty(\hvb x\sin\theta\cos\phi\\
    +\hvb y\sin\theta\sin\phi\\ +\hvb z\cos\theta)-\hvb x\cos\omega t-\hvb y\sin\omega t]\nonumber\\
    U&=-\vb m'\cdot\vb B-zMg,\ \ \vb m'=m'\qty(-\hvb x\sin\theta'\sin\phi'+\hvb y\sin\theta'\cos\phi'+\hvb z\cos\theta')\nonumber\\
    U&=-\frac{\mu_0mm'}{4\pi r^3}\mqty[-3\sin^2\theta\cos\qty(\phi-\omega t)\cos\phi\sin\theta'\sin\phi'+\cos\omega t\sin\theta'\sin\phi'\\
    +3\sin^2\theta\cos\qty(\phi-\omega t)\sin\phi\sin\theta'\cos\phi'-\sin\omega t\sin\theta'\cos\phi'\\
    +3\sin\theta\cos\theta\cos\qty(\phi-\omega t)\cos\theta']-r\cos\theta Mg\nonumber\\
    U&=-\frac{\mu_0mm'}{4\pi r^3}\mqty[3\sin^2\theta\sin\theta'\cos\qty(\phi-\omega t)\sin\qty(\phi-\phi')+\sin\theta'\sin\qty(\phi'-\omega t)
    \\+3\sin\theta\cos\theta\cos\qty(\phi-\omega t)\cos\theta']-r\cos\theta Mg\nonumber
\end{align}
A Lagrangiana é,
\begin{align}
    L&=\frac{M}{2}\mqty[\dot{r}^2\\+r^2\sin^2\theta\dot{\phi}^2\\+r^2\dot{\theta}^2]+\frac12\mqty[I_1\qty(\dot{\phi'}\sin\theta'\sin\psi'+\dot{\theta'}\cos\psi')^2\\
    +I_1\qty(\dot{\phi'}\sin\theta'\cos\psi'-\dot{\theta'}\sin\psi')^2\\+I_3\qty(\dot{\psi'}+\dot{\phi'}\cos\theta')^2]-U\nonumber
\end{align}

\subsection{Equações do movimento para $\psi'$}

\begin{align}
    \pdv{}{\psi'}L&=0\nonumber\\
    \pdv{}{\dot{\psi'}}L&=I_3\qty(\dot{\psi'}+\dot{\phi'}\cos\theta)=p_{\psi'}\nonumber\\
    \dv{}{t}p_{\psi'}&=0
\end{align}

\subsection{Equações do movimento para $\phi'$}

\begin{align}
    \pdv{}{\phi'}L&=-\frac{\mu_0mm'\sin\theta'}{4\pi r^3}\qty[3\sin^2\theta\cos\qty(\phi-\omega t)\cos\qty(\phi-\phi')-\cos\qty(\phi'-\omega t)]\nonumber\\
    \pdv{}{\dot{\phi'}}L&=I_1\dot{\phi'}\sin^2\theta'+p_{\psi'}\cos\theta'\nonumber\\
    I_1\ddot{\phi'}\sin^2\theta'&+2I_1\dot{\phi'}\dot{\theta'}\sin\theta'\cos\theta'-p_{\psi'}\sin\theta'\dot{\theta'}=\nonumber\\
    &-\frac{\mu_0mm'\sin\theta'}{4\pi r^3}\qty[3\sin^2\theta\cos\qty(\phi-\omega t)\cos\qty(\phi-\phi')-\cos\qty(\phi'-\omega t)]
\end{align}

\subsection{Equações do movimento para $\theta'$}

\begin{align}
    \pdv{}{\theta'}L&=\frac{\mu_0mm'}{4\pi r^3}\mqty[3\sin^2\theta\cos\theta'\cos\qty(\phi-\omega t)\sin\qty(\phi-\phi')+\cos\theta'\sin\qty(\phi'-\omega t)\\
    -3\sin\theta\cos\theta\sin\theta'\cos\qty(\phi-\omega t)]\nonumber\\
    &+I_1\dot{\phi'}^2\sin^2\theta'-\dot{\phi'}\sin\theta'p_{\psi'}\nonumber\\
    \pdv{}{\dot{\theta'}}L&=I_1\dot{\theta'}\nonumber\\
    I_1\ddot{\theta'}&=\pdv{}{\psi'}L
\end{align}

\subsection{Equações do movimento para $r$}

\begin{align}
    \pdv{}{r}L&=Mr\dot{\theta}^2+Mr\sin^2\theta\dot{\phi}^2+Mg\cos\theta-3\frac{\mu_0mm'}{4\pi r^4}\qty[\cdots]\nonumber\\
    \pdv{}{\dot{r}}L&=M\dot{r}\nonumber\\
    M\ddot{r}&=\pdv{}{r}L
\end{align}

\subsection{Equações do movimento para $\phi$}

\begin{align}
    \pdv{}{\phi}L&=\frac{3\mu_0mm'\sin\theta}{4\pi r^3}\qty[\sin\theta\sin\theta'\cos\qty(2\phi-\omega t-\phi')-\cos\theta\cos\theta'\sin\qty(\phi-\omega t)]\nonumber\\
    \pdv{}{\dot{\phi}}L&=Mr^2\sin^2\theta\dot{\phi}\nonumber
\end{align}

\subsection{Equações do movimento para $\theta$}

\begin{align}
    \pdv{}{\theta}L&=\frac{3\mu_0mm'\cos\qty(\phi-\omega t)}{4\pi r^3}\qty[\sin(2\theta)\sin\theta'\sin\qty(\phi-\phi')+\cos\qty(2\theta)\cos\theta']\nonumber\\
    &-r\sin\theta Mg+Mr^2\sin\theta\cos\theta\dot{\phi}^2\nonumber\\
    \pdv{}{\dot{\theta}}L&=Mr^2\dot{\theta}\nonumber
\end{align}

\section{Primeiras aproximações}

Para eliminar a dependência temporal explícita supomos que, 

\begin{align}
    \phi = \omega t+ \alpha\\
    \phi' = \omega t +\alpha'
\end{align}

\subsection{Equações do movimento para $\psi'$}

\begin{align}
    \pdv{}{\psi'}L&=0\nonumber\\
    \pdv{}{\dot{\psi'}}L&=I_3\qty(\dot{\psi'}+\omega\cos\theta)=p_{\psi'}\nonumber\\
    \dv{}{t}p_{\psi'}&=0
\end{align}

\subsection{Equações do movimento para $\phi'$}

\begin{align}
    \pdv{}{\phi'}L&=-\frac{\mu_0mm'\sin\theta'}{4\pi r^3}\qty[3\sin^2\theta\cos\qty(\alpha)\cos\qty(\alpha-\alpha')-\cos\qty(\alpha')]\nonumber\\
    \pdv{}{\dot{\phi'}}L&=I_1\omega\sin^2\theta'+p_{\psi'}\cos\theta'\nonumber\\
    2I_1\omega\dot{\theta'}\sin\theta'\cos\theta'&-p_{\psi'}\sin\theta'\dot{\theta'}=\nonumber\\
    &-\frac{\mu_0mm'\sin\theta'}{4\pi r^3}\qty[3\sin^2\theta\cos\qty(\alpha)\cos\qty(\alpha-\alpha')-\cos\qty(\alpha')]
\end{align}

\subsection{Equações do movimento para $\theta'$}

\begin{align}
    \pdv{}{\theta'}L&=\frac{\mu_0mm'}{4\pi r^3}\mqty[3\sin^2\theta\cos\theta'\cos\qty(\alpha)\sin\qty(\alpha-\alpha')+\cos\theta'\sin\qty(\alpha')\\
    -3\sin\theta\cos\theta\sin\theta'\cos\qty(\alpha)]\nonumber\\
    &+I_1\omega^2\sin^2\theta'-\omega\sin\theta'p_{\psi'}\nonumber\\
    \pdv{}{\dot{\theta'}}L&=I_1\dot{\theta'}\nonumber\\
    I_1\ddot{\theta'}&=\pdv{}{\theta'}L
\end{align}

\subsection{Equações do movimento para $r$}

\begin{align}
    \pdv{}{r}L&=Mr\dot{\theta}^2+Mr\sin^2\theta\omega^2+Mg\cos\theta-3\frac{\mu_0mm'}{4\pi r^4}\qty[\cdots]\nonumber\\
    \pdv{}{\dot{r}}L&=M\dot{r}\nonumber\\
    M\ddot{r}&=\pdv{}{r}L
\end{align}

\subsection{Equações do movimento para $\phi$}

\begin{align}
    \pdv{}{\phi}L&=\frac{3\mu_0mm'\sin\theta}{4\pi r^3}\qty[\sin\theta\sin\theta'\cos\qty(2\alpha-\alpha')-\cos\theta\cos\theta'\sin\qty(\alpha)]\nonumber\\
    \pdv{}{\dot{\phi}}L&=Mr^2\sin^2\theta\omega\nonumber
\end{align}

\subsection{Equações do movimento para $\theta$}

\begin{align}
    \pdv{}{\theta}L&=\frac{3\mu_0mm'\cos\qty(\alpha)}{4\pi r^3}\qty[\sin(2\theta)\sin\theta'\sin\qty(\alpha-\alpha')+\cos\qty(2\theta)\cos\theta']\nonumber\\
    &-r\sin\theta Mg+Mr^2\sin\theta\cos\theta\omega^2\nonumber\\
    \pdv{}{\dot{\theta}}L&=Mr^2\dot{\theta}\nonumber
\end{align}

\section{Aproximações seguintes}

Supomos que o segundo imã oscila quase exatamente abaixo do imã acima deste, isto é, 

\begin{align}
    \sin\theta \approx\theta
\end{align}

\subsection{Equações do movimento para $\psi'$}

\begin{align}
    \pdv{}{\psi'}L&=0\nonumber\\
    \pdv{}{\dot{\psi'}}L&=I_3\qty(\dot{\psi'}+\omega)=p_{\psi'}\nonumber\\
    \dv{}{t}p_{\psi'}&=0
\end{align}

\subsection{Equações do movimento para $\phi'$}

\begin{align}
    \pdv{}{\phi'}L&=-\frac{\mu_0mm'\sin\theta'}{4\pi r^3}\qty[3\theta^2\cos\qty(\alpha)\cos\qty(\alpha-\alpha')-\cos\qty(\alpha')]\nonumber\\
    \pdv{}{\dot{\phi'}}L&=I_1\omega\sin^2\theta'+p_{\psi'}\cos\theta'\nonumber\\
    2I_1\omega\dot{\theta'}\sin\theta'\cos\theta'&-p_{\psi'}\sin\theta'\dot{\theta'}=\nonumber\\
    &-\frac{\mu_0mm'\sin\theta'}{4\pi r^3}\qty[3\theta^2\cos\qty(\alpha)\cos\qty(\alpha-\alpha')-\cos\qty(\alpha')]
\end{align}

\subsection{Equações do movimento para $\theta'$}

\begin{align}
    \pdv{}{\theta'}L&=\frac{\mu_0mm'}{4\pi r^3}\mqty[3\theta^2\cos\theta'\cos\qty(\alpha)\sin\qty(\alpha-\alpha')+\cos\theta'\sin\qty(\alpha')\\
    -3\theta\sin\theta'\cos\qty(\alpha)]\nonumber\\
    &+I_1\omega^2\sin^2\theta'-\omega\sin\theta'p_{\psi'}\nonumber\\
    \pdv{}{\dot{\theta'}}L&=I_1\dot{\theta'}\nonumber\\
    I_1\ddot{\theta'}&=\pdv{}{\theta'}L
\end{align}

\subsection{Equações do movimento para $r$}

\begin{align}
    \pdv{}{r}L&=Mr\dot{\theta}^2+Mr\theta^2\omega^2+Mg-3\frac{\mu_0mm'}{4\pi r^4}\qty[\cdots]\nonumber\\
    \pdv{}{\dot{r}}L&=M\dot{r}\nonumber\\
    M\ddot{r}&=\pdv{}{r}L
\end{align}

\subsection{Equações do movimento para $\phi$}

\begin{align}
    \pdv{}{\phi}L&=\frac{3\mu_0mm'\theta}{4\pi r^3}\qty[\theta\sin\theta'\cos\qty(2\alpha-\alpha')-\cos\theta'\sin\qty(\alpha)]\nonumber\\
    \pdv{}{\dot{\phi}}L&=Mr^2\theta^2\omega\nonumber
\end{align}

\subsection{Equações do movimento para $\theta$}

\begin{align}
    \pdv{}{\theta}L&=\frac{3\mu_0mm'\cos\qty(\alpha)}{4\pi r^3}\qty[2\theta\sin\theta'\sin\qty(\alpha-\alpha')+\cos\theta']\nonumber\\
    &-r\theta Mg+Mr^2\theta\omega^2\nonumber\\
    \pdv{}{\dot{\theta}}L&=Mr^2\dot{\theta}\nonumber
\end{align}

\section{Aproximações seguintes}

Supomos que o segundo imã oscila quase exatamente em uma posição com valor $r = a$, isto é, 

\begin{align}
    r = a\qty(1+\epsilon),\ \ \epsilon\ll 1
\end{align}

\subsection{Equações do movimento para $\psi'$}

\begin{align}
    \pdv{}{\psi'}L&=0\nonumber\\
    \pdv{}{\dot{\psi'}}L&=I_3\qty(\dot{\psi'}+\omega)=p_{\psi'}\nonumber\\
    \dv{}{t}p_{\psi'}&=0
\end{align}

\subsection{Equações do movimento para $\phi'$}

\begin{align}
    \pdv{}{\phi'}L&=-\frac{3\mu_0mm'\sin\theta'}{4\pi a^3}\qty[\theta^2\cos\qty(\alpha)\cos\qty(\alpha-\alpha')+\epsilon\cos\qty(\alpha')-\frac13\cos\qty(\alpha')]\nonumber\\
    \pdv{}{\dot{\phi'}}L&=I_1\omega\sin^2\theta'+p_{\psi'}\cos\theta'\nonumber\\
    2I_1\omega\dot{\theta'}\sin\theta'\cos\theta'&-p_{\psi'}\sin\theta'\dot{\theta'}=\nonumber\\
    &-\frac{3\mu_0mm'\sin\theta'}{4\pi a^3}\qty[\theta^2\cos\qty(\alpha)\cos\qty(\alpha-\alpha')+\epsilon\cos\qty(\alpha')-\frac13\cos\qty(\alpha')]
\end{align}

\subsection{Equações do movimento para $\theta'$}

\begin{align}
    \pdv{}{\theta'}L&=\frac{3\mu_0mm'}{4\pi a^3}\mqty[\theta^2\cos\theta'\cos\qty(\alpha)\sin\qty(\alpha-\alpha')-\frac13\cos\theta'\sin\qty(\alpha')\\
    -\epsilon\cos\theta'\sin\qty(\alpha')-\theta\sin\theta'\cos\qty(\alpha)]\nonumber\\
    &+I_1\omega^2\sin^2\theta'-\omega\sin\theta'p_{\psi'}\nonumber\\
    \pdv{}{\dot{\theta'}}L&=I_1\dot{\theta'}\nonumber\\
    I_1\ddot{\theta'}&=\pdv{}{\theta'}L
\end{align}

\subsection{Equações do movimento para $r$}

\begin{align}
    \pdv{}{r}L&=Ma\dot{\theta}^2+Ma\theta^2\omega^2+Mg-3\frac{\mu_0mm'}{4\pi a^4}\qty(1-4\epsilon)\qty[\cdots]\nonumber\\
    \pdv{}{\dot{r}}L&=Ma\dot{\epsilon}\nonumber\\
    Ma\ddot{\epsilon}&=\pdv{}{r}L
\end{align}

\subsection{Equações do movimento para $\phi$}

\begin{align}
    \pdv{}{\phi}L&=\frac{3\mu_0mm'\theta}{4\pi a^3}\mqty[\theta\sin\theta'\cos\qty(2\alpha-\alpha')-\cos\theta'\sin\qty(\alpha)\\
    +3\epsilon\cos\theta'\sin\qty(\alpha)]\nonumber\\
    \pdv{}{\dot{\phi}}L&=Ma^2\theta^2\omega\nonumber
\end{align}

\subsection{Equações do movimento para $\theta$}

\begin{align}
    \pdv{}{\theta}L&=\frac{3\mu_0mm'\cos\qty(\alpha)}{4\pi a^3}\qty(1-3\epsilon)\qty[2\theta\sin\theta'\sin\qty(\alpha-\alpha')+\cos\theta']\nonumber\\
    &-a\qty(1+\epsilon)\theta Mg+Ma^2\qty(1+2\epsilon)\theta\omega^2\nonumber\\
    \pdv{}{\dot{\theta}}L&=Ma^2\qty(1+2\epsilon)\dot{\theta}\nonumber
\end{align}

\section{Aproximações seguintes}

Supomos que o segundo imã oscila com $\theta' \ll 1$, isto é,

\begin{align}
    \sin\theta'\approx\theta'
\end{align}

\subsection{Equações do movimento para $\psi'$}

\begin{align}
    \pdv{}{\psi'}L&=0\nonumber\\
    \pdv{}{\dot{\psi'}}L&=I_3\qty(\dot{\psi'}+\omega)=p_{\psi'}\nonumber\\
    \dv{}{t}p_{\psi'}&=0
\end{align}

\subsection{Equações do movimento para $\phi'$}

\begin{align}
    \pdv{}{\phi'}L&=\frac{\mu_0mm'\theta'\cos\alpha'}{4\pi a^3}\qty[1-3\epsilon]\nonumber\\
    \pdv{}{\dot{\phi'}}L&=I_1\omega\sin^2\theta'+p_{\psi'}\cos\theta'\nonumber\\
    2I_1\omega\dot{\theta'}\theta'&-p_{\psi'}\theta'\dot{\theta'}=\frac{\mu_0mm'\theta'\cos\alpha'}{4\pi a^3}\qty[1-3\epsilon]\nonumber\\
\end{align}

\subsection{Equações do movimento para $\theta'$}

\begin{align}
    \pdv{}{\theta'}L&=\frac{3\mu_0mm'}{4\pi a^3}\mqty[\theta^2\cos\qty(\alpha)\sin\qty(\alpha-\alpha')-\frac13\sin\qty(\alpha')\\
    -\epsilon\sin\qty(\alpha')-\theta\theta'\cos\qty(\alpha)]\nonumber\\
    &+I_1\omega^2{\theta'}^2-\omega\theta'p_{\psi'}\nonumber\\
    \pdv{}{\dot{\theta'}}L&=I_1\dot{\theta'}\nonumber\\
    I_1\ddot{\theta'}&=\pdv{}{\theta'}L
\end{align}

\subsection{Equações do movimento para $r$}

\begin{align}
    \pdv{}{r}L&=Ma\dot{\theta}^2+Ma\theta^2\omega^2+Mg-3\frac{\mu_0mm'}{4\pi a^4}\qty(1-4\epsilon)\qty[\theta'\sin\qty(\alpha')+3\theta\cos\alpha]\nonumber\\
    \pdv{}{\dot{r}}L&=Ma\dot{\epsilon}\nonumber\\
    Ma\ddot{\epsilon}&=\pdv{}{r}L
\end{align}

\subsection{Equações do movimento para $\phi$}

\begin{align}
    \pdv{}{\phi}L&=\frac{3\mu_0mm'\theta\sin\alpha}{4\pi a^3}\qty[3\epsilon-1]\nonumber\\
    \pdv{}{\dot{\phi}}L&=Ma^2\theta^2\omega\nonumber
\end{align}

\subsection{Equações do movimento para $\theta$}

\begin{align}
    \pdv{}{\theta}L&=\frac{3\mu_0mm'\cos\qty(\alpha)}{4\pi a^3}\qty[2\theta\theta'\sin\qty(\alpha-\alpha')+\cos\theta'-3\epsilon\cos\theta']\nonumber\\
    &-a\qty(1+\epsilon)\theta Mg+Ma^2\qty(1+2\epsilon)\theta\omega^2\nonumber\\
    \pdv{}{\dot{\theta}}L&=Ma^2\qty(1+2\epsilon)\dot{\theta}\nonumber
\end{align}

\end{document}
